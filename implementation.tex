\section{Implementation}
\label{sec-impl}

In \S \ref{sec-methods} the strategies and interfaces that \Cyclus uses to 
simplify archetype development were presented. These represent notions about
the amount of information and prior knowledge that the archetype developer 
must have in order to write archetypes.  If a particular particular strategy 
decreased the knowledge required by archetype developers then it is considered
easier and beneficial to implement.  

However, methods that are more inuitive for new users to understand are often
propotionatly more difficult to implement. For example, it is one thing to 
learn to play the game \emph{tic-tac-toe}, or even master it; it is quite another
thing to design the game \emph{tic-tac-toe} in the first place.  
A sublime interface belies 
herculean effort. This section descibes the infrastructre that holds up 
current \cyclus archetype development.  This is relevant to other fuel 
cycle simulators that wish to adopt the same strategies that \cyclus 
implements. In particular, the implemention of the \cyclus preprocessor, 
the type system, input file validation, metadata annotations will all 
be covered here.

\subsection{The \Cyclus Preprocessor}

The \cyclus preprocessor, \cycpp, is resposnible for all metadata collection and 
code generation for archetypes. It is implemented as a small Python utility 
currently less than 2000 lines in a single file.  It has no dependencies other 
than the Python standard library. It is thus light-weight enough to move around 
between code projects, if needed. For the scale of its responsibility, \cycpp
is extremely efficient. 

The preprocessor implements the three passes detailed in \S\ref{subsec-ppgc}:
normalization via standard \code{cpp}, annotation accumulation, and code 
generation. The \cycpp tool must be run on all C++ header and source files that
contain archetype code and the \code{#pragma cyclus} directives. Running \cycpp
on files without such directives does no harm and will result in exactly the 
original file. The first \cycpp
pass, running the C preprocessor over these files, is a trivial subprocess 
spawn. The only potential trouble spots here are ensureing that \cycpp sees the
same include, macro defintitions, and macro undefinitions that actual compilation 
of the source code will have.

\subsection{Database Types \& Backends}

\textbf{Robert C. and Anthony}

\subsubsection{Hashability}

\subsection{XML Validation}

\textbf{Katy}

\subsection{JSON Annotations}

\textbf{Radio}
