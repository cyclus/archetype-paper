\section{Implementation}
\label{sec-impl}

In \S \ref{sec-methods} the strategies and interfaces that \Cyclus uses to 
simplify archetype development were presented. These represent notions about
the amount of information and prior knowledge that the archetype developer 
must have in order to write archetypes.  If a particular particular strategy 
decreased the knowledge required by archetype developers then it is considered
easier and beneficial to implement.  

However, methods that are more inuitive for new users to understand are often
propotionatly more difficult to implement. For example, it is one thing to 
learn to play the game \emph{tic-tac-toe}, or even master it; it is quite another
thing to design the game \emph{tic-tac-toe} in the first place.  
A sublime interface belies 
herculean effort. This section descibes the infrastructre that holds up 
current \cyclus archetype development.  This is relevant to other fuel 
cycle simulators that wish to adopt the same strategies that \cyclus 
implements. In particular, the implemention of the \cyclus preprocessor, 
the type system, input file validation, metadata annotations will all 
be covered here.

\subsection{cycpp}

\textbf{Anthony and Matt}

\subsection{Database Types \& Backends}

\textbf{Robert C. and Anthony}

\subsubsection{Hashability}

\subsection{XML Validation}

\textbf{Katy}

\subsection{JSON Annotations}

\textbf{Radio}
