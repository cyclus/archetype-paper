\section{Motivations \& Problem Statement}
\label{sec-motive}

The agent is the fundamental abstraction in Agent-based modeling frameworks 
such as \cyclus. However, many problems that are 
solved with an \gls{ABM} approach are sufficiently represented by one or two 
agents \cite{taylor2014agent}. However, in nuclear fuel cycle simulations there is
a proliferation of facility types that are distinct both conceptually and in the 
kinds of physical processes that they represent. While it would be \emph{possible}
to merge all facility types into a single, highly parameterized agent model,
it would be \emph{unwise} to do so. Facilities that model fundamentally different
physics should not be combined into a single class. Not only is it more work to 
combine them but it is also less intelligible. The same reasoning applies to why it is 
not a good idea to merge the concepts of institutions and regions with that of
facilities; the subject of the model is fundamentally distinct from other models
and this separation of concerns should be maintained.

So unlike other agent-based frameworks, the modularity of \cyclus drives 
an ecosystem of \emph{archetypes}. An archetype is an agent class that specifies 
how the agent should behave via its own implementation of physics, chemistry, 
economic, and social policies. Archetypes are parameterized only in ways that 
make sense to the policies that they implement. Extraneous policies are left to 
other archetypes. For example, a nuclear reactor would not be parameterized based 
on separation efficiencies capacity, that would be left for a reprocessing facility.

All archetypes are agents. They are able to communicate (through resource 
exchange) with all other agents and they have access to the same information 
about the environment in which they live. The archetype abstraction provides
speciation of agents so that each archetype may fill its own fuel cycle niche.

Archetypes are an essential abstraction layered on top of the agent abstraction. 
Fuel cycle facility, region, and institution modelers should directly create
archetypes rather than raw agents. Thus for people trying to create, use, and 
extend \cyclus agents, archetypes should be their entry and exit point. Because this 
concept is central to how \cyclus works, such users are known as \emph{archetype
developers}.

The archetype abstraction has the added advantage that archetype developers 
need only be specialists in the field of the archetype that they are developing.
Someone who knowledgeable about gaseous centrifuges could design 
an enrichment facility. A person that studies deep geologic repositories
could model a long-term storage facility. A reactor physicist could create 
a suite of archetypes for various reactor technologies. In this way, an ecosystem 
of archetypes from representative experts can be built up. Since one does not need
to be an expert to use an archetype, a well-developed ecosystem will provide 
a huge benefit to everyone. The separation of concerns provided by the archetype 
abstraction maximizes quality over the entire fuel cycle modeling process.

Still, for any simulator to be successful its key abstractions must be both easily 
configured by users and easily modified by developers.  If these activities are 
too difficult, the barrier to entry for new users and developers will be 
insurmountable in a reasonable time frame. Otherwise, potential new users and developers will 
walk away in confusion and frustration. For \cyclus, archetype development
needs to have first-class support.

This paper discusses how the \cyclus code base has overcome the inherent limitations
of its design goals in order 
to provide a fertile platform on which to model the nuclear fuel cycle.
Such strategies can apply to any and all agent-based fuel cycle simulators, of which 
there is currently only \cyclus. It can work not only on nuclear fuel cycle but it can work on
other types of fuel cycle as well, but that will require some extra work and time.  
