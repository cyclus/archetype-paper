\section{Motivations \& Problem Statement}
\label{sec-motive}

Agent-based modeling frameworks necessarily place the agent as the fundamental 
abstraction. This is true in \cyclus as well. However, many problems which are 
solved with an \gls{ABM} approach are sufficiently represented by one or two kinds of 
agents \citeme. However, in nuclear fuel cycle simulations we know that there is
a proliferation of facility types that are distinct both conceptually and in the 
kinds of physical processes that they implement. While it would be \emph{possible}
to merge all facility types into a single, highly-parameterized agent abstraction
it would be \emph{unwise} to do so. Facilities that model fundamentally different
physics should not be combined into a single class. Not only is it more work to 
do so but it is also less intelligible. The same reasoning applies to why it is 
not a good idea to merge the concepts of institutions and regions with that of
facilities; the subject of the model is fundamentally distinct form other models
and this separation of concerns should be maintained.

So unlike other agent-based frameworks, the modularity of \cyclus drives 
an ecosystem of \emph{archtypes}. An archetype is an agent class that specifies 
how the agent should behave via its own implementation of physics, chemistry, 
economic, and social policies. Archtypes are parameterized only in ways that 
make sense to the policies that they implement. Extraneous policies are left to 
other archetypes. For example, a nuclear reactor would not be parameterized based 
on separation efficiencies capacity, that would be left for a reprocessing facility.

All archetypes are agents. They are able to communicate (through resource 
exchange) with all other agents and they have access to the same information 
about the environment in which they live. The archetype abstraction provides
speciation of agents so that each archetype may fill its own fuel cycle niche.

Archetypes are an essential abstraction in addition to the agent abstraction. 
Fuel cycle facility, region, and institution modelers should directly create
archetypes rather than raw agents. Thus for people trying to create, use, and 
extend \cyclus agents, archetypes should be their entry and exit point. Because this 
concept is central to how \cyclus works, such users are known as \emph{archetype
developers}.

The archetype abstraction has the added advantage that archetype developers 
need only be specialists in the field of the archetype that they are developing.
Someone who knows everything about gaseous centrifuges could design 
an enrichment facility. Someone who knows all about deep geologic repositories
could model a long-term storage facility. A reactor physicist could create 
a suite of archetypes for various reactor technologies. In this way, an ecosystem 
of archetypes from representative experts can be built up. Since you do not need
to be an expert to use an archetype, a well developed ecosystem would provide 
a huge benefit to everyone. The separation of concerns provided by the archetype 
abstraction maximizes quality over the entire fuel cycle modeling process.

Still, for any simulator to be successful its key abstractions must be easily 
configured by users and and easily modified by developers.  If these activities are 
too difficult, the barrier to entry for new users and developers will be 
insurmountable in a reasonable time frame. Potential new users and developers will 
walk away in confusion and frustration. For \cyclus, archetype development
needs to have first-class support.

An informal study 
by the developers at the time showed that a new developer circa \cyclus v0.1
took 2+ work weeks to get a working `do-nothing' archetype running. This is obviously 
too long because most researchers do not have two weeks of time `just to try 
something out.' By \cyclus v0.3 the do-nothing development time had been reduced 
to approximately one business week. In \cyclus v0.4 this time became about 3 days.
As of \cyclus v1.0, this finally was reduced down to 1 - 4 hours, which meets
appropriate expectations for someone attempting cyclus out as a first time archetype 
developer.

These dramatic development time reductions were caused be two forces:
clarification of the archetype abstraction and explicit tools to help with 
archetype creation. Initially the region-institution-facility hierarchy 
had been firmly established. How this hierarchy fit into an agent-based 
paradigm took much longer to fully illustrate. Once the notion was fleshed out,
\Cyclus should make creation of archetypes as easy as possible. 
For a long time in the history of this simulator and its predecessors, such as 
\gls{GENIUSv2} \todo{CITE}, tools to help make archetypes were notable by their abscense. 
The addition of such tools by the \cyclus core developers made archetype
developers significantly more efficient.

The archetype development tools provided by \cyclus must overcome a vareity of 
technical hurdles in order to make archetype developmement accessible.  These 
include but are not limited to the lack of reflection in the C++ programming language, 
the desire to support multiple database formats, automatic validation of input files,
special mechanisms for handling resource exchange and inventory persistence, 
and the somewhat complex interface required to support the snappshotting and 
restart of simulations. Such concerns are basic to \cyclus operation but 
ancillary to the physics, chemistry, and economics being modeled by an archetype.
The minutia should `just work' since it is not a core part of a fuel cycle model.

This paper shows how the \cyclus code base overcame these inherent limitations
and thereby provided a fertile platform on which to model the nuclear fuel cycle.
Such strategies apply to any and all agent-based fuel cycle simulators, of which 
there is currently only \cyclus.
