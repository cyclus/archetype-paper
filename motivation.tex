\section{Motivations \& Problem Statement}

Agent-based modeling frameworks necessarily place the agent as the fundemtnal 
abstraction. This is true in \cyclus as well. However, many problems which are 
solved with an ABM approach are sufficiently reprersented by one or two kinds of 
agents \citeme. However, in nuclear fuel cycle simulations we know that there is
a proliferation of facility types that are distinct both conceptually and in the 
kinds of physical processes that they implement. While it would be \emph{possible}
to merge all facility types into a single, highly-parameterized agent abstraction
it would be \emph{unwise} to do so. Facilities that model fundementally different
physics should not be combined into a single class. Not only is it more work to 
do so but it is also less intelligible. The same reasoning applies to why it is 
not a good idea to merge the concepts of institutions and regions with that of
facilitities; the subject of the model is fundementally distinct form other models
and this separation of concerns should be maintained.

So unlike other agent-based frameworks, the modularity of \cyclus drives 
an ecosystem of \emph{archtypes}. An archetype is an agent class that specifies 
how the agent should behave via its own implementation of physics, chemisity, 
economic, and social policies. Archtypes are parameterized only in ways that 
make sense to the policies that they implement. Extraneous policies are left to 
other archetypes. For example, a nuclear reactor would not be parameterized based 
on separation efficiencies capacity, that would be left for a reprocessing facility.

All archetypes share agenthood. They are able to communicate (through resource 
exchange) with all other agents and they have access to the same information 
about the environment in which they live. The archetype abstraction provides
speciation of agents so that each archetype may fill its own fuel cycle niche.

Archetypes are an essential abstraction in addition to the agent abstraction. 
Fuel cycle facility, region, and institution modelers should directly create
archetypes rather than raw agents. Thus for people trying to use and extend 
\cyclus agents, archetypes should be their entry and exit point. Because this 
concept is central to how \cyclus works, such users are known as \emph{archetype
developers}.


Establishment of Archetype developers as first-class target

Need for usable tools

Heuristic of developer time for first archetype

Creates a framework for usable developers to be experts in specific facility 
technologies but not the fuel cycle while reaping the benefit of experts in 
other technologies and the fuel cycle as a whole.

Overcome technical issues such as the last of reflection in C++, multiple 
database formats, validation, etc.

Complex APIs to support restart
