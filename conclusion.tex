\section{Conclusions}

Writing archetypes can be a daunting task because reasonably accurate models 
involve physics, economics, and computer science to solve a single nuclear engineering 
problem.  Unlike other spheres of nuclear engineering, decoupling these domains from
one another is often not possible without significant simplification. \cyclus is 
no exception to this and is designed to allow for complete fidelity throughout 
all aspects of the simulation. However, enabling full modeling fidelity does
not imply that archetype developers need to actively address this every class of 
problem every time they pursue a new archetype.

\Cyclus succeeds in simplifying archetype developement by identifying a category 
of computer science \& software developement problems that are addressed 
algorithmically. This moves effort away from humans, who are pursing physics and
economics, and onto computers. This automation happens by default for as much of
an archetype as possible. In this case the archetype developer writes approximately
$10\times$ less software, excluding physics and resource exchange routines, than
they would manually. However the automation may be cherry-picked piece by piece,
down to nothing if desired. The \cyclus system of archetype development nibmly 
adapts to the needs of the archetype by allowing the developer to select the 
the level of computer science that they explitily use.  

The \cyclus system makes for better fuel cycle simulations by creating better 
archetypes.  This is because of two consequences of the preprocessor. The first 
\cyclus encorages developers to write the archetype
that they intended to implement. The second is that the archetypes are automatically
validated.

State variables are easy to create. When a software feature has a high cost to use,
developers will minimize the number of times that they invoke it. This can 
sometimes lead to sacraficing model fidelity in an effort to author more concise
code, which is its own virtue. However, the long-term completeness of an archtype
with repscet to its physics calculations should not be based, even in part, on the
short-term impetus to have a minimum-viable product.

Further imporvements.