\section{Conclusions}
\label{sec-conc}

Writing archetypes can be a daunting task because reasonably accurate models 
require knowledge of physics, economics, and computer science to solve a single nuclear engineering 
problem.  Unlike other spheres of nuclear engineering, decoupling these domains from
one another is often not possible without significant simplification. \cyclus is 
no exception to this and is designed to allow for complete fidelity throughout 
all aspects of the simulation. The advantage of the \cyclus design is that it 
enables full modeling fidelity without requiring that archetype developers actively 
address every class of problem every time they pursue a new archetype.

\Cyclus succeeds in simplifying archetype development by identifying a category 
of computer science and software development problems that are addressed 
algorithmically. This moves effort away from humans, who are pursing physics and
economics, and onto computers. This automation happens by default for
archetype development and reduces manual code writing by approximately
10 times.
Additionally, the automation may be partially or fully reduced 
if the default generated code is not desired.

Formula
 
\% difference =((After - Before)/Before) *100
 
The number of lines of code saved by using the preprocessor is shown in Table
\ref{lines}.

\begin{table}
\caption{ Number of lines saved from Archetypes in Cyclus v1.3.1}
\label{lines}
\begin{tabular}{|lrrl|}
\toprule
file &  before &  after & \% difference \\
\hline
deploy\_inst.cc   &      70 &     70 &         0.00 \\
\hline
deploy\_inst.h    &      85 &    268 &       215.29 \\
\hline
enrichment.cc    &     465 &    465 &         0.00 \\
\hline
enrichment.h     &     371 &    608 &        63.88 \\
\hline
fuel\_fab.cc      &     715 &    715 &         0.00 \\
\hline
fuel\_fab.h       &     256 &    647 &       152.73 \\
\hline
growth\_region.cc &     160 &    160 &         0.00 \\
\hline
growth\_region.h  &     128 &    283 &       121.09 \\
\hline
manager\_inst.cc  &      94 &     94 &         0.00 \\
\hline
manager\_inst.h   &      63 &    155 &       146.03 \\
\hline
reactor.cc       &     502 &   1392 &       177.29 \\
\hline
reactor.h        &     387 &    255 &        34.11 \\
\hline
separations.cc   &     288 &    289 &         0.35 \\
\hline
separations.h    &     199 &    491 &       146.73 \\
\hline
sink.cc          &     182 &    347 &        90.66 \\
\hline
sink.h           &     136 &    137 &         0.74 \\
\hline
source.cc        &     117 &    127 &         8.55 \\
\hline
source.h         &     115 &    223 &        93.91 \\
\bottomrule
\end{tabular}
\end{table}


The \cyclus system enables better fuel cycle simulations by creating better 
archetypes.  Improved archetypes are a direct consequence of two features of the preprocessor. First, 
\cyclus encourages developers to write the archetype
that they intended to implement. Secondly, the archetypes are automatically
validated.

State variables are easy to create. When a software feature has a high cost to use,
developers will minimize the number of times that they invoke it. This can 
sometimes lead to sacrificing model fidelity in an effort to author more concise
code. However, the long-term completeness of an archetype
with respect to its physics calculations should not be based, even in part, on the
short-term impetus to have a minimum-viable product. By dramatically reducing the 
length of time it takes to implement a state variable, archetype developers implement
more state variables and thus more precise and tunable models.

Furthermore, automatically generating archetype code removes typographic errors and 
\cyclus interface misuse. This avoids potential and frustrating problems in 
archetype development. The 
generated code derives its validity from \cycpp, which itself is extensively 
tested. Any errors accidentally introduced by \cycpp would be endemic to all archetypes, 
but a fix to \cycpp would be the corresponding solution. Archetypes are thus 
partially vetted due to \cycpp.

The preprocessor also generates schema for archetypes. This provides a mechanism 
for automatically validating user input.  Without validation, the archetype 
developer has no guarantee that the user has entered a meaningful or physically possible 
value (e.g., negative fluxes). Rather than approaching this problem in an
\emph{ad hoc} 
manner, the \cyclus interface demands that user input be examined via \gls{RNG}.  The overhead 
of this requirement is mitigated since the archetype developer obtains the schema
for free. This assures a high level of quality in using archetypes as well as 
developing them.

The strategies detailed and implemented in this paper radically
reduce the overhead of writing archetypes. By enabling more expressibility and greater
modifiability, developers and simulators are able to more easily experiment.  
Alternative fuel cycle representations may be explored quantitatively at an
unparalleled rate.

However, archetype development tools and approaches are not without further 
potential refinements. \cycpp will undergo continued improvement 
as more archetypes are developed and common usage patterns emerge. Codifying and 
auto-generating these patterns is a rich area of exploration. The authors anticipate 
that inventory and resource exchange patterns will be among the first
targeted, 
which would likely take the form of new filters in pass three.

Furthermore, the fundamentals of the \cyclus type system will be used in 
perpetuity, as enabled by the extensibility of the type system.  More types
will be added as needed by the archetype developers.  It is also possible to 
make the type system dynamic, allowing for custom types to be implemented at run time,
which will reduce the run time as well.
This could be a boon to archetype developers seeking to create a wide variety of custom
tables.

The preprocessor could also improve the generated code. \Cyclus reserves the right
to add additional metadata keys and associated meanings.  For example, a \code{'range'}
key could specify the acceptable range for a state variable. This could in turn 
provide better validation by adding bounds checks beyond the what is performed
by \gls{RNG}. A similar strategy could be followed for categorical variables whereby
set membership would be verified. The code that is generated for future keys 
depends largely on the meaning ascribed to those keys. There is no limit 
to the richness of available metadata.

In summary, archetype development is has been made significantly easier. This is due in large
part to the advent of the \cyclus preprocessor. While \cycpp itself is the fruit of
a large computer science and software development effort, it is used here 
primarily to improve fuel cycle simulations. \cyclus provides a solid ground 
as a platform for archetype development while simultaneously nimble enough
to allow for future growth. Independent of \cyclus, the strategies and methods that
\cyclus implements for archetype developers are translatable to any agent-based 
fuel cycle simulator.  Some aspects, such as the type system,  may even be exportable 
to general simulation science. 

