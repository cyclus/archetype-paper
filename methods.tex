\section{Methods \& Strategies}
\label{sec-methods}

Archetype development can be a daunting task on its own. For this reason, 
many fuel cycle simulators choose to supply only a limited corral of 
pre-built archetypes. This places the responsibility of creating an maintaining 
the archetypes with the authors of the simulator.  When only pre-built archetypes are available, then this suite defines the scope 
of all possible fuel cycles that a user can model. If the scope is not broad 
enough for the needs of a user, then the simulator developer must expand the 
scope or lose the user.  This can create a bottleneck because the number of users 
may increase while the productivity of the simulation developer remains constant.

\Cyclus avoids such bottlenecks by empowering users to create and maintain 
their own archetypes independent of the development timeline of the \cyclus 
kernel. However, this modularity comes with its own costs. Being able to 
plug in user-created archetypes implies an \gls{API} \sout{where} \comment{so that (?)} 
the archetype conforms to certain well-defined expectations. This is an 
additional burden to both nascent and experienced archetype developers that 
does not effect the underlying behavior or physics. Rather, a large portion 
of archetype code exists to satisfy the \cyclus \gls{API}. \comment{unclear transition}

Therefore, mitigating the difficulties in writing archetypes has been a major effort 
for \cyclus. This includes overcoming the hurdle of interfacing with 
\cyclus itself and, to the extent possible, providing tools that 
ease the implementation of the physics and behavior desired. Thus, 
a variety of strategies are used by \cyclus to ease development: 

\begin{itemize}
    \item \textbf{Preprocessing \& Code Generation:}  By automatically 
        inspecting and creating portions of archetypes, the overhead
        of adhering to the \cyclus \gls{API} is removed. This also 
        adds limited reflection to C++ as needed, which is important 
        for archetypes that wish to know about themselves \comment{access metadata?}.

    \item \textbf{Type System Abstraction:} The \Cyclus type system 
        provides a common basis for archetypes to store and retrieve 
        complex types in the database. This alleviates the need for archetype
        developers to invent and implement custom persistence solutions.

    \item \textbf{Validation:} Archetypes use \gls{XML} schema to validate that 
        the prototypes that are generated from them are created correctly.
        This ensures the users of an archetype are using it correctly.

    \item \textbf{Metadata Annotations:} Archetypes have a standard place to 
        store and retrieve both pre-defined and arbitrary metadata.
        This allows for archetype developers to communicate relevant information
        to tools outside of \cyclus (such as a visualization tool) without
        losing any information in \comment{the process of being an archetype. (awk)}

    \item \textbf{Package System:} \Cyclus has a packaging system for archetypes and 
        libraries of archetypes. This means that all archetype developers can 
        uniquely specify their own archetypes without the fear of overlapping 
        names.  For example, two developers could each have a \texttt{Reactor}
        archetype, but they would exist in different packages and be 
        disambiguated.

    \item \textbf{Removal of Markets:} In an agent-based methodology, 
        the mechanism for communication between agents is not an agent itself.
        Thus in \cyclus, market resolution was moved to be solely in the purview 
        of kernel. In order to transfer resources, agents communicate 
        in a well-defined way. Thus, archetype development need not  
        include markets that specify how archetypes wish to communicate. \comment{Check this edit!} 

\end{itemize}

Though the above mechanisms are discussed with respect to \cyclus, they are 
transferable to any agent-based modeling framework that requires modular agent 
archetypes. The following subsections present greater detail regarding these 
strategies.

\subsection{Preprocessing \& Code Generation}
\label{subsec-ppgc}

Every \cyclus archetype is required to implement the member functions 
seen in Listing \ref{req-api} and may optionally implement those seen in 
Listing \ref{opt-api}. Due to object orientation in 
C++ and how \cyclus stores \sout{state}\comment{states?}, these member functions must be implemented directly
on the archetype itself. The implementation that archetypes inherit 
from the \code{Agent} class is not sufficient. 

\begin{lstlisting}[caption={Required Archetype API}, label=req-api]
void InfileToDb(InfileTree* tree, DbInit di);
void InitFrom(cyclus::QueryableBackend*);
void InitInv(cyclus::Inventories& invs);
cyclus::Inventories SnapshotInv();
void Snapshot(cyclus::Agent*);
std::string schema();
Json::Value annotations();
cyclus::Agent* Clone();
void InitFrom(cyclus::Agent*);
\end{lstlisting}

\begin{lstlisting}[caption={Optional Archetype API}, label=opt-api]
void Build(cyclus::Agent* parent);
void EnterNotify();
void BuildNotify(cyclus::Agent* child);
void Decommission(cyclus::Agent* child);
\end{lstlisting}

Archetypes store state\comment{s?} as public or private member variables
directly on the class.  For example, a \code{Reactor} archetype class
could have a burnup declared as `\code{double burnup};'. Not every agent has 
a burnup (e.g., \code{Enrichment}) and so the \code{burnup} member 
should not be part of \code{Agent}. Furthermore, archetypes are modularly 
defined. Because \cyclus cannot know all possible field names 
for all possible archetypes for all time, it can only react to how 
archetypes are written. In a dynamic language, the \code{Agent}
superclass would still be able to to \emph{inspect} instances of its subclasses
to discover field names at runtime. Such introspection is called \emph{reflection}. 
This would allow \code{Agent} to hold a single, generic implementation of each of the
member functions in Listing \ref{req-api}, relieving archetype developers from 
having to implement them manually.

However, C++ lacks reflection. Neither \code{Agent} nor archetypes are allowed
to dynamically discover what their member variable names are at runtime.  This 
implies that the archetype developer must explicitly code in the appropriate variable
names in the correct way for each of the nine required functions.  For instance, 
the \code{Reactor} class must explicitly save the \code{burnup} member in the 
\code{Snapshot()} function if \code{burnup} is to be saved to the database.
Nothing comes along for free and minor typos can cause large breakages. \comment{(want to reword this sentence but not sure how to)}

Since implementing this part of the archetype \gls{API} is both highly error-prone and 
routine, it is ripe for automatic code generation. Code generation replaces the 
tedious task of writing the required member functions with software that will 
insert such member functions into an otherwise fully developed archetype. 

The code generation strategy has some limitations \comment{'requirements' seems to be a more accurate word here?}. First, it must be 
performed to generate valid C++ prior to compilation \comment{(check edit)}. Second, the
code generator must be provided with enough information about the archetype in 
order to accurately create the function implementation. Third and finally, it 
is highly desirable to keep archetypes in valid C++ \comment{depending on target journal, you may want to define this concept (the reader perhaps assuming that invalid code would simply not run)}. Templating languages 
such as Jinja \TODO{CITE} or any other variety of custom solutions would allow for 
expressive code generation. However, such template languages would be yet another 
tool for the archetype developer to learn, running counter to the goal of 
simplifying development.

The limitations above are all elegantly addressed through the use of a 
\cyclus-aware preprocessor. The first stage of C/C++ compilation is the 
C Prepocessor or \code{cpp} \TODO{CITE}. This tool is responsible for expanding 
\code{#include} directives and implementing other \code{#} directives
\comment{, such as pragma}. It is executed
prior to any other stage of compilation (lexing, parsing, tokenizing, \emph{etc.}).
Importantly, the \code{#pragma} directive is skipped by \code{cpp} if it is not 
recognized and the directive \comment{is retained its output (awk)}.
\comment{Does this work? "Importantly, the pragma directive is skipped by \code{cpp} if it is not recognized, so the directive retains its output."} It is a
purposeful hook for other preprocessors to use and implement their own code generation.
If an alternative preprocessor only uses \code{#pragma} directives as its interface, 
the developer will be able to write in pure C++ and reap the benefits
of an extra code-generation step. 

However, all code generators must be supplied with sufficient information about
where and how to create the code. These tasks may also be accomplished through 
the use of pragmas. To handle a suite of such utilities, a custom preprocessor
is needed.

Certain pragmas may be used to parse only the needed information 
about an archetype, rather than parsing the entire class. For an archetype, the 
member variables that are saved and loaded from its instances are the most important \sout{.This is because it is} \comment{because} they \sout{that the} fully describe the state of an agent 
at all points in the simulation. These member variables are known as \emph{state variables}. 
To generate the appropriate I/O routines for an archetype, at a minimum, the names and C++ types 
of all state variables must be known. The gathering \comment{is this an OK word to use?} of all these state variables is is 
called \emph{state accumulation}. 

\comment{Still, other pragmas may 
signal where to insert automatically generated code. The tedious 
part of the \code{cyclus::Agent} interface should be created. \emph{<---I do not understand this sentence.} Specifically, 
the member functions in Listing \ref{req-api} should be inserted automatically
so that archetype developer need not write them.}

The \cyclus preprocessor, called \cycpp, handles all of the cyclus code generation
and state accumulation. This tool only recognizes directives that begin with \code{#pragma cyclus} so as to uniquely distinguish it from other preprocessors.
\cycpp performs three main passes on archetype code:
\begin{enumerate}
    \item Normalization of the source via the standard preprocessor \code{cpp} 
    \item Accumulation of state and other agent annotations from normalized code 
    \item Code generation into original source
\end{enumerate}
Importantly, any preprocessor that adds reflection must make at
minimum two passes: discovery of what exists (state accumulation) and adding this
information back to the class (code generation). Single-pass preprocessors such 
as the standard \code{cpp} utility are not sufficient to enable reflection because they 
cannot guarantee that all relevant class information has been seen when code generation begins. \sout{by the point 
where code generation must happen.} Thus, \cycpp includes these two passes plus an additional 
initial pass to simplify state accumulation. Further passes could be added that 
implement reflection onto the generated code itself, but this is often unnecessary. 
Particularly in \cyclus, further passes would be excessive since the portions of the 
\code{cyclus::Agent} interface that are generated are known ahead of time.

\cyclus-specific pragmas are broken up into two categories depending on whether 
they are most relevant to the state accumulation or code generation steps. They 
are denoted as \emph{annotation directives} and \emph{code generation directives},
respectively.

The \cycpp annotation interface has two main directives:
\begin{itemize}
    \item \code{#pragma cyclus var <dict>} - state variable annotation
    \item \code{#pragma cyclus note <dict>} - agent annotation or note
\end{itemize}
The state variable annotation is used on the line immediately above an archetype
member variable to declare it as a state variable. The note directive is used
anywhere in an archetype class declaration and applies to the archetype itself.
Both of these have a \code{<dict>} argument that is a Python dictionary. This 
holds metadata about the state variable or the archetype. For example, a \code{flux}
state variable declaration is shown in Listing \ref{flux-pragma}.

\begin{lstlisting}[caption={Flux State Variable Annotation}, label=flux-pragma]
#pragma cyclus var {"default": 42.0, "units": "n/cm2/s"}
double flux;
\end{lstlisting}

The code generation interface in its simplest form occurs via the \cyclus 
prime directive, \code{#pragma cyclus}.  This expression will 
generate the entire archetype interface and insert it in place of the pragma.
More fine-grained code generation may be used by passing additional arguments.
The signature for such targeted code generation may be seen in Listing \ref{targ-cg}.
The first argument is one of \code{decl}, \code{def}, or \code{impl} representing 
interface declarations, definitions (the declaration and the implementation together), 
or implementations (just the body without the declaration) of the archetype member
functions, respectively. Lacking any further arguments, this will produce the desired code for 
all of member functions in Listing \ref{req-api}.  Optionally, a member function 
name in all lowercase (i.e., \code{inittodb} rather than \code{InitToDb}) may 
be supplied to generate code for the given function alone.  Lastly, in 
cases where \cycpp cannot determine the archetype to apply the code generation for,
such as an in an implementation file (\code{*.cc} or \code{*.cpp}), the agent
class name may be provided as a final parameter.

\begin{lstlisting}[caption={Targeted Code Generation Directive Signatures}, 
                   label=targ-cg]
#pragma cyclus <decl|def|impl> [<func> [<agent>]]
\end{lstlisting}

\sout{When the annotation and code generation directives are combined and expanded with 
\cycpp, simple agents become simple to write and complex agents confine their 
complexity to the physics and economics that they seek to implement.} \comment{When the annotation and code generation directives are combined and expanded with 
\cycpp, agents become simple to write and complexity is confined to their physics and economics.} The base \cyclus
infrastructure is largely removed from the concern of the archetype developer.
For example, a simple reactor model may be completely implemented as seen in 
Listing \ref{rx-eg}. The barrier to creating new archetypes is much lower than 
compared to the hundreds of lines of code that this would require without \cycpp.

\begin{lstlisting}[caption={Simple Reactor Archetype}, label=rx-eg]
class Reactor : public cyclus::Facility {
 public:
  Reactor(cyclus::Context* ctx) {};
  virtual ~Reactor() {};

  #pragma cyclus

 private:
  #pragma cyclus var {'default': 4e14, 'units': 'n/cm2/2'}
  double flux;

  #pragma cyclus var {'default': 1000, 'units': 'MWe'}
  float power;

  #pragma cyclus var {'doc': 'Are we operating?'}
  bool shutdown;
};
\end{lstlisting}

\subsection{Type System Abstraction}

Related to the issue of how to snapshot and restart simulations is the issue of 
what fundamental data types are allowed to be saved and loaded natively by the 
simulation. For many physics simulators, primitive data types (i.e., \code{float}, 
\code{int}, and \code{string}) and arrays of these types are sufficient to 
express and evaluate the underlying equations. This handful of types is small enough 
that most simulators can handle them in an ad hoc manner. Furthermore, these types
often align with the structure of most databases, allowing for easy translation
between disk and memory.

Agent-based fuel cycle modeling, however, subsumes physical modeling frameworks.
The types of data that are needed to naturally express many fuel cycle concepts 
do not fit into the standard `arrays of floats' mindset.  More sophisticated types are needed,
such as maps and sets of primitive types \sout{at a minimum}. For example, 
a reprocessing facility needs to have a mapping from elements to separation 
efficiencies. To avoid hard-coding the elements into the archetype, the user should 
be able to state which elements are allowed for each separation. The correct data
type in C++ for such information is \code{std::map<int, double>}. Any other 
representation, say an \code{int array} or a \code{float array},  would likely be converted to 
a \code{map} by the archetype itself.

\Cyclus increases expressiveness of archetypes and reduces error from extraneous 
transcription by natively supporting its own extensible type system. Every type
has its own unique integer identifier \sout{that distinguishes it }as well as 
a variable name corresponding to this type. All types
must have a C++ representation, which is used for the type of the state
variables. \comment{Is this correct? "The types of the state variables are given by corresponding C++ representations."} Every database format may then choose which types in the type system it 
supports and how it implements them. All types have a static \emph{rank}, or the 
number of variable length dimensions, they support.  For example, \code{int} and 
\code{double} are both rank-0, while \code{vector<float>} is rank-1 due to 
vectors having arbitrary length. As an optimization, the archetype developer 
may also give a \emph{shape}, a maximum size along each dimension, for a state 
variable. The type system is extensible \sout{in that new types may be added in the future}, 
allowing for expressiveness to grow and evolve with the needs of archetype developers.

This strategy represents a significant abstraction over the needs and usage of most 
other simulator technologies. This level of detail is required by \cyclus due to the
dynamically loadable agents. Without a strong type system, state variables 
would only be minimally useful and archetypes would have to rely on out-of-simulator
mechanisms to save and load their state.

\subsection{Metadata Annotations}

Metadata annotations are coupled with code generation.  State variable annotations
are provided by using the \code{#pragma cyclus var} directive and annotations for 
the archetypes as a whole are given using the \code{#pragma cyclus note} directive,
as specified in \S \ref{subsec-ppgc}. Both of these directives take a dictionary 
as an argument and this mapping must have string keys.

Metadata serves an important purpose by communicating information to the \cyclus
preprocessor, to the \cyclus kernel, and even beyond the simulation to analysis 
and visualization tools. Some metadata is automatically generated from the 
archetype declaration itself.  This information is considered read-only and required. 
Other annotations are supplied by the archetype developers and are optional, 
though for some keys (such as documentation) they are highly recommended.

Any key may be supplied to the annotation directives. For most keys, the 
purpose of the entry is given entirely by the archetype developer. However, 
some keys are reserved and have special meanings in various contexts. Table
\ref{sv-anno} lists these keys for state variable annotations and Table
\ref{ag-anno} displays the reserved keys for \code{#pragma cyclus note}.

\begin{table}
\caption{Special State Variable Annotations}
\begin{tabular}[htbc]{|p{0.2\linewidth}|p{0.75\linewidth}|}
\hline
\textbf{key} & \textbf{meaning} \\
\hline
\code{type}    & The C++ type. \emph{Read-only}.\\
\hline
\code{index}   & State variable order-of-appearence, 0-indexed. \emph{Read-only}.\\
\hline
\texttt{default} & The default value for this variable that is used if otherwise 
                 unspecified by the user. The type of the value must match the 
                 type of the variable.\\
\hline
\code{shape}   & The shape of the data type. If present this must
                 be a list of integers of a given length (rank).
                 Specifying positive values will, depending on the 
                 backend, render this a fixed-length data type
                 of the provided length. A value of \code{-1}  
                 will retain the variable-length along that axis. 
                 Fixed-length variables are normally more performant and thus it is 
                 often better practice to specify a shape if possible. For 
                 example, a length-5 string would have a shape of \code{[5]} and 
                 a length-10 vector of variable-length strings would have a 
                 shape of \code{[10, -1]}. \comment{to shorten this, you could
nix the last sentence}\\
\hline
\code{doc}     & Documentation string.\\
\hline
\code{tooltip} & Brief documentation string for user interfaces.\\
\hline
\code{units}   & The physical units, if any, as a string.\\
\hline
\code{userlevel} & Integer (0 - 10) representing ease (0) or difficulty (10) 
                   in using this variable, default 0.\\
\hline
\code{schematype} & The data type that is used in the schema for input file
                    validation. This enables the user to supply just the data type
                    rather than having to overwrite the full schema for this state
                    variable. \sout{In most cases - when the shape is rank 0 or 1 such
                    as for scalars or vectors - this is simply a string. In cases
                    where the rank is 2+ this is a list of strings.} Please refer to
                    the \gls{XML} Schema Datatypes \cite{xml-datatypes}
                    for more information.\\
\hline
\code{initfromcopy} & Code string to use in the \code{InitFrom(Agent* m)} 
                      function for the state variable instead of automatic code 
                      generation.\\
\hline
\code{initfromdb} & Code string to use in the 
                    \code{InitFrom(QueryableBackend* b)} 
                    function for this state variable instead of automatic code 
                    generation.\\
\hline
\code{infiletodb} & Code strings to use in the \code{InfileToDb()} function 
                    for this state variable instead of automatic code generation.
                    This is a dictionary of string values with the keys `read'
                    and `write' that represent reading values from the input file 
                    writing them out to the database, respectively.\\
\hline
\code{schema}  & Code string to use in the \code{schema()} function for 
                 this state variable instead of automatic code generation.
                 This is an \gls{RNG} string. This is usually coupled with 
                 \code{infiletodb} to ensure the custom
                 schema is read into the database correctly.\\
\hline
\code{snapshot} & Code string to use in the \code{Snapshot()} function for 
                  this state variable instead of automatic code generation.\\
\hline
\code{snapshotinv} & Code string to use in the \code{SnapshotInv()} function 
                     for this state variable instead of automatic code generation.\\
\hline
\code{initinv} & Code string to use in the \code{InitInv()} function for 
                 this state variable instead of automatic code generation.\\
\hline
\end{tabular}
\label{sv-anno}
\end{table}

\begin{table}
\caption{Special Agent Archetype Annotations}
\begin{tabular}[htbc]{|p{0.2\linewidth}|p{0.75\linewidth}|}
\hline
\textbf{key} & \textbf{meaning}\\
\hline
\code{vars} & The state variable annotations. \emph{Read-only}.\\
\hline
\code{name} & C++ class name (string) of the archetype. \emph{Read-only}.\\
\hline
\code{entity} &  String of the type of archetype based on which class it 
                 inherits from; \code{cyclus::Region},
                 \code{cyclus::Institution}, or \code{cyclus::Facility}are
                 given by `region', `institution', or `facility',
                 respectively. If the class inherits from \code{cyclus::Agent} but 
                 not the previous three 
                 the string `archetype' is used. The string 'unknown' is used
                 if the class does not inherit from \code{cyclus::Agent}.
                 \emph{Read-only}.\\
\hline
\code{parents} & List of string class names of the direct super-classes of this
                 archetype. \emph{Read-only}.\\
\hline
\code{all_parents} & List of string class names of all the super-classes of this
                     archetype. \emph{Read-only}.\\
\hline
\code{doc} & Documentation string.\\
\hline
\code{tooltip} & Brief documentation string for user interfaces.\\
\hline
\code{userlevel} & Integer (0 - 10) representing ease (0) or 
                   difficulty (10) in using this variable, default 0.\\

\hline
\end{tabular}
\label{ag-anno}
\end{table}

With respect to \cycpp, metadata enables the customization of code generation
without the need to alter the preprocessor itself or create an entirely new 
code generator. This tailoring of \cycpp is exemplified by keys, such as \code{'shape'}, 
\code{'schematype'}, and \code{'initfromcopy'}.  Still, other keys are generated by \cycpp itself
and are considered read-only.
These include the all-important \code{'type'}, \code{'index'}, \code{'name'}, and 
\code{'parents'} keys.  \comment{Do you need the quotes around the keys?} The automatic
creation of these keys minimizes human transcription error for the most important 
metadata. This is a core simplification of archetype development.

Metadata is similarly important for the \cyclus kernel itself. All metadata is 
directly available via the \code{annotations()} member function, which returns 
a \gls{JSON} 
object (equivalent to a Python dictionary with string keys). Combined with the 
auto-generated read-only keys from \cycpp, the metadata provides much needed 
reflection to the archetypes.  Unlike most C++ classes, archetypes have 
runtime access to their own class names, their parent classes, and the names and 
types of their state member variable\comment{s?}. Therefore, the archetype may 
make runtime decisions about how to behave based on how it is defined.
The major current use for the limited reflection in archetypes is for agents to 
save and load themselves. However, other uses are anticipated. \comment{This
last sentence should probably either be expanded upon or deleted?}

Lastly, metadata is useful beyond \cyclus itself. Annotation keys such as `doc',
`tooltip', and `units' are included to provide end-user documentation. The `userlevel'
and other keys signal how archetypes and state variables should be treated in 
downstream user interfaces \comment{should these keys be formatted as above,
using the code command? AO addition: Above is code with quotes, here is just quotes,
and tables use code only. I vote code only for all because it looks better!}.  Even keys that might have a primary usage elsewhere, 
such as for `default' values, may still be helpful beyond the scope of \cyclus 
itself.  

The var and note\comment{(formatting)} directives provide unambiguous locations for implementing and 
generating metadata annotations. Metadata being contained completely within archetype 
declarations increases the worth of both the metadata and the archetypes
because there is a single source for information about archetypes.
Additionally, the high-degree locality eases the creation of archetypes,
which is in turn amplified by the automatic discovery of the annotation 
keys. \comment{or: Additionally, this high degree of locality eases the creation of
archetypes, enabled by the automatic discovery of the annotation keys.}

\subsection{Validation}

When writing \cyclus input files, users configure archetypes into prototypes 
for the simulation.  Configuration is done by assigning all state variables to 
  initial values. This is the principal mechanism by which information flows 
  from users to archetype developers. Therefore, it is reasonable for archetype 
  developers to ask, ``Does what the user gave me make sense?'' For example: 

\begin{itemize} 
    \item a flux state variable should not be negative
    \item an integer variable should not receive the string ``Toaster''
    \item a vector representing an N-group cross-section should have exactly N 
          elements
\end{itemize} 
If the user were to not follow these rules, the input file would be break physical
constraints, type constraints, or shape constraints, respectively. 
Such breaking of constraints must be an error and the simulation should not be 
allowed to run. 

Therefore, these rules can be codified into the archetypes allowing 
the simulator to check that whether or not any input file adheres to them prior
to execution. This is known as \emph{validation} and \cyclus implements it 
automatically to the extent possible. \comment{I thought the beg of this
paragraph fit with the previous paragraph.}

\Cyclus input files are written in \gls{XML}, \sout{. \gls{XML} itself} \comment{which}  may be validated against a 
known structure called a \emph{schema}. The schema is itself \gls{XML} that describes 
the layout, types, and attributes of the input it will validate.
\Cyclus provides a default schema that describes the overall structure of the 
input file.

Archetypes, however, are dynamically loaded and so schema for their state 
variables may not be predicted or preloaded. To ensure consistency, archetype schema 
must be loaded along with the archetype itself.  While tempting to ignore the 
notion of archetype schema and not validate the state variables, this would
create a system where effectively no contract exists between the 
user and the archetype developer. Even if an archetype developer implemented 
\emph{ad hoc} validation to their own classes, users would not expect
\comment{anticipate?} such 
restrictions.  To avoid such a morally precarious position, \cyclus requires
that archetypes provide their own validation via the \code{schema()} member 
function seen in Listing \ref{req-api}.

That said, schemas are two steps removed from the C++ that the archetypes
are written in: \gls{XML} itself and the schema interface. \comment{I'm not
sure how to reword this but I think it should be somehow. My attempt: To accomplish the
above, the schemas must be written in \gls{XML} to satisfy the schema
interface, which is two steps removed from the C++ that the
archetypes are written in.} This provides additional 
cognitive load to the archetype developer as they now must learn two additional
tools before writing an archetype. However, the schema for most state
variables may be derived automatically from information known to the preprocessor:
the name of the state variable, its type, and optionally its shape and size in 
the case of containers. Thus the \code{schema()} member function is auto-generated
by \cycpp and input validation is obtained for free. 

\gls{XML}-based schema are extraordinarily useful as a mechanism for validating types. 
For example, if the user provides a floating point number rather than an 
expected string, they should be alerted to this error immediately.
Moreover, schema can also validate structure to assert that 
a variable has the right shape. A length-5 vector must be initialized with 
five elements. However, rules about type and shape are at the limit of what may 
be auto-generated. \comment{This last sentence here seems out of place to me.
I'm sure this is based in my lack of familiarity with the subject matter, but
I'm assuming it either needs more of a transition or it is unnecessary and
can be deleted. Actually, it may fit better as the intro sentence to the next
paragraph?}

Semantic and physical meaning must be ascribed by the archetype developer.
Giving meaning to state variables based
on their name and C++ type alone is impossible to accomplish via an anutomated
method. \comment{Check edit} For example, that a variable named \code{flux} 
on a \code{Reactor} should not be negative comes from a physical understanding
and not a computational one. Such meaning can and should be given by the 
archetype developer via metadata.

Metadata for physical validation amounts to modifications to the schema. There are 
two annotation keys that may be used to accomplish this.  The first clarifies the 
type that the schema uses and is called \code{'schematype'}.  \comment{In the example of a group structure shown in Listing \ref{ngroups}, the \code{'schematype'} could be assigned a value of 
\code{'positiveInteger'} rather than relying on the default \code{'int'} type
that permits negatives and zeros.} \sout{This example may be seen in Listing \ref{ngroups}. }
Additionally, the metadata key \code{'schema'}
can be used to wholly replace the auto-generated schema for the state variable at 
hand. \comment{!ensure consistent key formatting!} This key can be used to change the name of the state variable with respect 
to the schema or to make a variable optional \comment{without providing a default value 
and other less dangerous operations (awk)}.

\begin{lstlisting}[caption={Physical Constraint Addition via `schematype'}, label=ngroups]
#pragma cyclus var {'schematype': 'positiveInteger'}
int ngroups;
\end{lstlisting}

Finally, \Cyclus also allows for the circumvention of automatic schema generation. This 
provides another method for instilling semantic meaning into state variable annotations
that does not rely on metadata annotations, although
partial or complete circumvention of code generation will require
more effort from the archetype developer. The fine-grained control afforded by 
hand-writing the \code{schema()} member function is performed 
by advanced developers and only when absolutely necessary due to insufficiencies 
in \cycpp.

\subsection{Package System}

In a robust ecosystem of archetypes, it is nearly guaranteed that different archetype
developers will want to use the same name. No single person or organization can 
reasonably lay sole claim to generic terms such as \emph{reactor}, \emph{source}, 
\emph{sink}, etc \sout{and a plethora of other such names}. Simultaneously, the archetype developer should not 
be concerned with accidental name collision between their archetypes and archetypes
of other
past, present, and future developers.  Furthermore, it is often useful in a 
simulation or 
development campaign to group similar or related archetypes together. Uniqueness
and collection problems are simultaneously solved through a hardy \emph{package system}.

\cyclus packaging is an organizational structure that defines where on the file system 
archetypes are installed to, how the \cyclus kernel will load installed
packages, and how to uniquely identify an archetype in an input file.  Archetypes 
are denoted with a three-part \emph{archetype specification}. When spelled out, this
is a colon-separated string with the following elements:
\begin{enumerate}
    \item a slash-separated (\texttt{/}) directory path,
    \item a library name, and
    \item an archetype name.
\end{enumerate}
For example, \code{my/path:mylib:MyAgent} represents \code{MyAgent} living
in \code{mylib} residing in the \code{'my/path'} directory. More than just 
a simple spelling convention, this is a necessary
tool for searching for and discovering archetypes on the machine of the user.

The path portion of the specification is relative to the \code{CYCLUS_PATH}. This is 
an list of directories on the machine that will be searched in order to find the 
archetype of interest. By default, \code{CYCLUS_PATH} contains the current working 
directory, the \cyclus install directory, and the \cyclus build directory. 
\code{CYCLUS_PATH} may also be modified as an environment variable, allowing the user
to permanently or temporarily alter the \cyclus search behavior.  Thus the path 
specification (e.g. \code{'my/path'}) is interpreted as a sub-directory of all of 
the directories on the \code{CYCLUS_PATH}. For a directory \code{d1} on 
\code{CYCLUS_PATH}, if  \code{'d1/my/path'} does not exist then the search for 
the archetype will continue along with \code{d2}, and so on. The path portion 
may be an empty string, indicating that the library lives directly on the 
\code{CYCLUS_PATH}.

The library name is the dynamically loadable library file name that stores the 
archetype. This does not include the \code{lib-} prefix or the file extension, 
which is generally operating system-dependent.  For example, on a
posix \comment{(should this include the acronym for POSIX?)} system,
a file named \code{libmyagents.so} would receive the library name \code{myagents}
in the archetype specification. 
If a library name is not specified, then it is assumed to be the same as the 
archetype name.
If desired, a single path may hold many librarys and a single library may hold
many archetypes. Thus, archetypes may be grouped
together coarsely or finely, depending on the needs of the archetype developers.

The path and library names together allow for complete disambiguation of 
archetypes because they enforce an important degree of namespacing.  It is unlikely
that two well-designed libraries will overlap in both library and archetype name.
Even if they overlap, one or both libraries may be placed in respective
sub-directories and the path is used to establish uniqueness.
This strategy for specifying archetypes ultimately removes confusion and error from 
both archetype developers and users alike.

\subsection{Stock Resource Exchange Algorithms}

In early versions of \cyclus, the dynamic resource exchange algorithm that the 
kernel used was itself dynamically loadable. Such an algorithm was called a
\emph{market} and was categorized as an entity alongside regions,
institutions, and facilities. 
Each commodity was traded in its own market, which was specified by 
the user in the input file.

Unlike the other entities, though, a market did not have agency.  It could not 
communicate with other agents in the simulation because it was itself the method
for agent communication.  Furthermore, dynamic resource exchange is the keystone
part of all fuel cycle simulators. \comment{I'm not sure if I'm
misunderstanding this sentence, but is this statement accurate?} Relegating such algorithms to live outside of 
the kernel lead to maintenance problems 
ensuring that all markets correctly supported the proper exchange interface,
thus minimizing the value of modularity within /cyclus.

Therefore, the notion of markets as a simulation entity was removed. In their 
stead, dynamic resource exchange algorithms were brought into the core to
guarantee exchange feasibility. Moreover, this enabled all commodities to trade 
through a single global exchange. The commodity itself automatically defines 
the sub-exchange graph, which may be thought of as analogous to the old 
markets, in which the commodity participates.\comment{I don't think I
understand this sentence enough to reword it. Can someone else do so?}

Since markets did not initially have any agency, their removal did not 
affect the agent-based
nature of \cyclus.  Rather, \comment{plug-in??} market removal allowed archetypes to 
communicate through a common resource exchange interface.

In summary, the current interpretation of dynamic resource exchange 
eases the burden on archetype developers. Because the primary duty of the 
kernel is to provide generic and valid resource exchange algorithms,
the archetype developer is not required to construct a custom exchange for 
each commodity an archetype trades. Furthermore, market removal from the
kernel does not 
impinge on exchange solver availability or customization.  Many exchanges may 
be provided via user-tuneable parameters.  The only restriction 
is that the exchange algorithms must \comment{already??} exist within \cyclus itself.  This is not 
considered overly burdensome, because individuals seeking to write custom 
exchanges - arguably the most advanced task in \cyclus - have likely 
transitioned from being an archetype developer to also being a kernel developer.
