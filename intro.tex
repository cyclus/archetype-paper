\section{Introduction}
\label{sec-intro}

\Cyclus \cite{cyclus_v1_0,cyclus_v1_2,huff_fundamental_2016} is the first truly agent-based
\cite{jennings2000agent} nuclear fuel cycle simulator.
New technologies, while exciting, often pose unforeseen challenges.
\Cyclus is no exception to this rule.  This paper answers the questions,
\emph{``What precisely is an agent in a nuclear fuel cycle context?''} and
\emph{``What features of simulation can be abstracted away to ease the burden
on nuclear fuel cycle researchers?''}

The genesis of \cyclus lies in the desire to
treat mass balances as discrete, model facilities individually rather than as
a fleet, and to be able to quantitatively compare the effects of changing the
fidelity of the facilities themselves. These goals necessitate a large degree of
sophistication on the part of the simulator infrastructure.  Resource exchange
must be handled in a generic and dynamic way as opposed to being hard-coded
for specific commodities. Simulations must be inherently comparable, which involves
storage infrastructure that is designed around this need. Agents must be able
to communicate with one another and learn about the environment in which they
exist. Additionally, agents must be able to be dynamically deployed and the set of
available agent models may not be collected until run time.

What sets nuclear fuel cycle simulation apart from traditional agent-based simulations
is the high degree of agent specialization? Standard agent-based simulators
are characterized as having a few types of agents, often only one and
almost always ($<5) less than five \cite{taylor2014agent}. The agent types are then
specialized
when they are instantiated \emph{in situ}. This model is insufficient for
nuclear fuel cycle applications.  For example, it would be unwise to have a single facility model
that represents both enrichment and reactors, called \code{EnrichmentOrReactor},
that decides via a switch how it behaves when it is deployed. It is much
more natural to have two models, \code{Enrichment} and \code{Reactor},
that implement their own physics calculations independently.

\Cyclus enables agent specialization along two separate axes. The first defines
an agent as an \emph{entity} that determines its role in the
nuclear fuel cycle. There are three kinds of entities:\emph{regions} that
represent geographic and governmental concerns,\emph{institutions}
that manage other agents, and \emph{facilities} that implement
physics calculations and are usually in charge of resource management.

The other axis of agent specialization distinguishes among who
writes the model, who sets up the model for potential use in a simulation,
and who actually deploys concrete representations of the model.
At the highest level are \emph{archetypes} whose behavior is parameterizable.
Archetypes are software implementations
of physical, chemical, economic, and political models.
For example, a \code{Reactor}  archetype may be
parameterized by a target burnup. Authors of these highly reusable models
are known as \emph{archetype developers}. Archetypes are in turn
\emph{configured} into \emph{prototypes}. A prototype is a copy
of the archetype but with all parameterizations set to concrete
values. Hence, a \code{Reactor} with a burnup of 42 MWd/kg is a
prototype. Configuring archetypes into prototypes is done by the \cyclus user
in the input file. Configuration requires no underlying knowledge of
how the archetype is implemented, though that often helps.
Finally, when prototypes are copied and \emph{deployed} in the simulation
they become \emph{agents}. This usage of the term `agent' to mean
the \emph{in situ} object is consistent with other agent-based
literature \cite{macal2005tutorial}.  Agent deployment happens exclusively via
\cyclus itself; manual deployment of agents is not allowed.
Archetypes that wish to deploy agents must \emph{schedule} their building
and decommissioning.

Archetype development is one of the most difficult aspects of agent-based
simulation. Physical models must be determined, implemented, and
validated. Moreover, this is where agents interface with the
\cyclus kernel itself. The dynamism of agent-based simulation
coupled with the need to have traditional simulator features
(such as restart and validation) creates a complex interface with the kernel.
Archetype development would be much simplified if saving and
loading from disk and communicating with other agents were
never a concern, though such a simulator would be of marginal use.
Due to these complexities, archetype development in \cyclus
has historically been difficult. Obtaining a working archetype
that performed no physics calculations from scratch would
take novice developers upwards of two weeks effort.  The complexities
that lead to such a high bar are emblematic not just of \cyclus,
but of any agent-based nuclear fuel cycle simulator.

Informal polling by the \cyclus team over the years
showed that a new developer circa \cyclus v0.1
took 2+ weeks to get a working `do-nothing' archetype. This is obviously
too long because most researchers do not have two weeks of time `just to try
something out.' By \cyclus v0.3 the do-nothing development time had been reduced
to approximately one business week. In \cyclus v0.4, this time became about 3 days.
As of \cyclus v1.0, this finally was reduced down to 1 - 4 hours, which meets
appropriate expectations for someone attempting \cyclus as a first time archetype
developer.


\begin{tabular} {|r|r|}
\hline
Version & Time to run \\
\hline
\cyclus v0.1 & 2+week \\
\hline
\cyclus v0.3 & 5 days \\
\hline
\cyclus v0.4 & 3 days \\
\hline
\cyclus v1.0 & 1-4 hours \\
\hline
\end{tabular}

Two forces caused these dramatic development time reductions:
first,clarification of the archetype abstraction and second,  explicit tools to help with
archetype creation. While the region-institution-facility hierarchy was established
early on, the formulation of this hierarchy in an agent-based paradigm took much
longer to firmly establish.  Once this notion had been refined, attention turned to
simplifying \Cyclus archetype development.
For a long time in the history of this simulator and its predecessors, such as
\gls{GENIUSv2} \cite{oliver_studying_2009}, tools to help make archetypes were
absent.
The addition of archetype tools described in this paper made development significantly more efficient.

The archetype development tools provided by \cyclus must overcome a variety of
technical hurdles in order to make archetype development accessible.  These
include, but are not limited to, the lack of reflection in the C++ programming language,
the desire to support multiple database formats, automatic validation of input files,
special mechanisms for handling resource exchange and inventory persistence,
and the somewhat complex interface required to support the snapshotting and
restart of simulations. Such concerns are basic to \cyclus operation but
ancillary to the physics, chemistry, and economics being modeled by an
archetype developer.
The minutia should `just work' since it is not a core part of a fuel cycle model.

This paper describes the strategies, efforts, interfaces,
and implementations that considerably reduce the complexity
an archetype developer must deal with directly. This has been
found to reduce development effort for simple agents down to a couple
of hours for novices. Expert archetype developers realize further
reductions down to a couple of minutes. This has has been accomplished
without sacrificing an iota of simulation fidelity. Since such
problems proliferate among agent-based simulators, the methods described here apply beyond \cyclus.
Though such methods sometimes dive into
computer science and agent-based modeling details, they are always implemented
with the express goal of making fuel cycle simulation
simpler, easier, faster, more expressive. 

This paper proceeds by first providing a more detailed motivation in \S
\ref{sec-motive}. Then, \S \ref{sec-methods} describes both high-level design
strategies and software interfaces used to ease archetype development and
provide the correct fuel cycle agent abstractions; a detailed discussion of the
underlying algorithms is provided as a supplement to the reader in \S
\ref{sec-impl}.  Lastly, \S \ref{sec-conc} provides final remarks and
illustrates potential future directions for \cyclus archetype development.
