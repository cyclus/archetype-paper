\section{Introduction}
\label{sec-intro}

\Cyclus \citeme is the first truly agent-based \citeme fuel cycle simulator. 
New technologies, while exciting, often pose new and unforeseen challenges.
\Cyclus is no exception to this rule.  This paper answers the questions,
\emph{``What precisely is an agent in a fuel cycle context?''} and 
\emph{``What is the best way to program an agent?''}

A discrete time agent-based simulator is needed to \comment{sate the want (awkward)} for increased 
fuel cycle modeling fidelity. The genesis of \cyclus lies in the desire to 
treat mass balances \comment{(but didn't we discuss that cyclus doesn't really do mass balance?)} as discrete, model facilities individually rather than as 
a fleet, and to be able to quantitatively compare the effects of changing the 
fidelity of the facilities themselves. These goals imply a large degree of 
sophistication on the part of the simulator infrastructure.  Resource exchange
must be handled in a generic and dynamic way as opposed to being hard-coded 
for specific commodities. Simulations must be inherently comparable, which involves 
storage infrastructure that is designed around this need. Agents must be able 
to communicate with one another and learn about the environment in which they 
exist. Additionally, agents must be able to be dynamically deployed and the set of 
available agent models may not be collected until run time. Building and using such a
simulator, while difficult, is not nearly as Sisyphean as it might seem. 

What sets fuel cycle simulation apart from traditional agent-based simulations 
is the high degree of agent specialization. Standard agent-based simulators
are characterized as having few types of agents, often only one and 
almost always less than five \citeme. The agent types are then specialized 
when they are instantiated \emph{in situ}. This model is not appropriate for 
fuel cycle applications.  For example, it would be unwise to have a single facility model 
that represents both enrichment and reactors, called \code{EnrichmentOrReactor},
that decides via a switch how it behaves when it is deployed. It is much 
more natural to have two models, \code{Enrichment} and \code{Reactor}, 
that implement their own physics calculations independently.

\Cyclus enables agent specialization along two separate axes. The first defines 
an agent as an \emph{entity} that determines its role in the 
fuel cycle. There are three kinds of entities: \emph{regions} that 
represent geographic and governmental concerns, \emph{institutions} 
that manage other agents, and \emph{facilities} that implement 
physics calculations and are usually in charge of resource management.

The other axis of agent specialization distinguishes among who 
writes the model, who sets up the model for potential use in a simulation,
and who actually deploys concrete representations of the model.
At the highest level are \emph{archetypes}, which are software implementations
of physical, chemical, economic, and political models, whose behavior
is parameterizable. For example, a \code{Reactor} \comment{extraneous arrow after 'reactor'?} archetype may be 
parameterized by a target burnup. Authors of these highly reusable models 
are known as \emph{archetype developers}. Archetypes are in turn 
\emph{configured} into \emph{prototypes}. A prototype is a copy 
of the archetype but with all parameterizations set to concrete 
values. Hence, a \code{Reactor} with a burnup of 42 MWd/kg is a 
prototype. Configuration is often specified by the \cyclus user 
in the input file. Configuration requires no underlying knowledge of 
how the archetype is implemented, though that often helps.
Finally, when prototypes are copied and \emph{deployed} in the simulation 
they become \emph{agents}. This usage of the term `agent' to mean 
the \emph{in situ} object is consistent with other agent-based
literature \citeme.  Agent creation happens exclusively via 
\cyclus itself; manual deployment of agents is not allowed.
Archetypes that wish to deploy agents must \emph{schedule} their building
and decommissioning.

Within the domain of agent specialization, archetype development is certainly the most taxing.
Not only must physical models be determined, implemented, and 
validated, but this is also where agents interface with the 
\cyclus kernel itself. The dynamism of agent-based simulation 
coupled with the compulsion to have traditional simulator features, 
such as restart, engenders a complex interface with the kernel.
Archetype development would be much simplified if saving and 
loading from disk and communicating with other agents were 
never a concern, though such a simulator would be of marginal use.
Due to these complexities, archetype development in \cyclus
has historically been difficult. Obtaining a working archetype
that performed no physics calculations from scratch would 
take novice developers upwards of two weeks effort.  The complexities
that lead to such a high bar are emblematic not just of \cyclus,
but of any agent-based fuel cycle simulator.

This paper describes the strategies, efforts, interfaces
and implementations that considerably reduce the complexity  
an archetype developer must deal with directly. This has been 
found to reduce development effort for simple agents down to a couple 
of hours for novices. Expert archetype developers realize further 
reductions down to a couple of minutes. This has has been accomplished 
without sacrificing an iota of simulation fidelity. Since such 
problems proliferate throughout all possible simulators in 
the \cyclus category, the methods described here apply beyond \cyclus.
Though such methods sometimes dive into the minutia of 
computer science and agent-based modeling, they are always implemented
with the express goal of making fuel cycle simulation 
simpler, easier, faster, more expressive, and validated.

This paper proceeds by first providing a more detailed motivation 
in \S \ref{sec-motive}. Then, \S \ref{sec-methods} describes 
methods and mindsets that are used to ease archetype development.
Some of these are high-level design strategies, while others are software interfaces that 
provide the correct fuel cycle agent abstractions. 
\S \ref{sec-impl} describes the underlying algorithms and usages
for those methodologies, which also require concrete implementations.
Lastly, \S \ref{sec-conc} provides final remarks and illustrates
potential future directions for \cyclus archetype development.
