
\section{Metadata Tables}
\label{sec-meta-tbl}


\begin{table}
\caption{Special State Variable Annotations}
\begin{tabular}[h!tbc]{|p{0.2\linewidth}|p{0.75\linewidth}|}
\hline
\textbf{key} & \textbf{meaning} \\
\hline
\code{type}    & The C++ type. \emph{Read-only}.\\
\hline
\code{index}   & State variable order-of-appearence, 0-indexed. \emph{Read-only}.\\
\hline
\texttt{default} & The default value for this variable that is used if otherwise 
                 unspecified by the user. The type of the value must match the 
                 type of the variable.\\
\hline
\code{shape}   & The shape of the data type. If present this must
                 be a list of integers of a given length (rank).
                 Specifying positive values will, depending on the 
                 backend, render this a fixed-length data type
                 of the provided length. A value of \code{-1}  
                 will retain the variable-length along that axis. 
                 Fixed-length variables are normally more performant and thus it is 
                 often better practice to specify a shape if possible. For 
                 example, a length-5 string would have a shape of \code{[5]} and 
                 a length-10 vector of variable-length strings would have a 
                 shape of \code{[10, -1]}.\\
\hline
\code{doc}     & Documentation string.\\
\hline
\code{tooltip} & Brief documentation string for user interfaces.\\
\hline
\code{units}   & The physical units, if any, as a string.\\
\hline
\code{userlevel} & Integer (0 - 10) representing ease (0) or difficulty (10) 
                   in using this variable, default 0.\\
\hline
\code{schematype} & The data type that is used in the schema for input file
                    validation. This enables the user to supply just the data type
                    rather than having to overwrite the full schema for this state
                    variable. In most cases - when the shape is rank 0 or 1 such
                    as for scalars or vectors - this is simply a string. In cases
                    where the rank is 2+ this is a list of strings. Please refer to
                    the \gls{XML} Schema Datatypes \cite{xml-datatypes}
                    for more information.\\
\hline
\code{initfromcopy} & Code string to use in the \code{InitFrom(Agent* m)} 
                      function for the state variable instead of automatic code 
                      generation.\\
\hline
\code{initfromdb} & Code string to use in the 
                    \code{InitFrom(QueryableBackend* b)} 
                    function for this state variable instead of automatic code 
                    generation.\\
\hline
\code{infiletodb} & Code strings to use in the \code{InfileToDb()} function 
                    for this state variable instead of automatic code generation.
                    This is a dictionary of string values with the keys `read'
                    and `write' that represent reading values from the input file 
                    writing them out to the database, respectively.\\
\hline
\code{schema}  & Code string to use in the \code{schema()} function for 
                 this state variable instead of automatic code generation.
                 This is an \gls{RNG} string. This is usually coupled with 
                 \code{infiletodb} to ensure the custom
                 schema is read into the database correctly.\\
\hline
\code{snapshot} & Code string to use in the \code{Snapshot()} function for 
                  this state variable instead of automatic code generation.\\
\hline
\code{snapshotinv} & Code string to use in the \code{SnapshotInv()} function 
                     for this state variable instead of automatic code generation.\\
\hline
\code{initinv} & Code string to use in the \code{InitInv()} function for 
                 this state variable instead of automatic code generation.\\
\hline
\end{tabular}
\label{sv-anno}
\end{table}

\begin{table}
\caption{Special Agent Archetype Annotations}
\begin{tabular}[h!tbc]{|p{0.2\linewidth}|p{0.75\linewidth}|}
\hline
\textbf{key} & \textbf{meaning}\\
\hline
\code{vars} & The state variable annotations. \emph{Read-only}.\\
\hline
\code{name} & C++ class name (string) of the archetype. \emph{Read-only}.\\
\hline
\code{entity} &  String of the type of archetype based on which class it 
                 inherits from; \code{cyclus::Region},
                 \code{cyclus::Institution}, or \code{cyclus::Facility}are
                 given by `region', `institution', or `facility',
                 respectively. If the class inherits from \code{cyclus::Agent} but 
                 not the previous three 
                 the string `archetype' is used. The string 'unknown' is used
                 if the class does not inherit from \code{cyclus::Agent}.
                 \emph{Read-only}.\\
\hline
\code{parents} & List of string class names of the direct super-classes of this
                 archetype. \emph{Read-only}.\\
\hline
\code{all_parents} & List of string class names of all the super-classes of this
                     archetype. \emph{Read-only}.\\
\hline
\code{doc} & Documentation string.\\
\hline
\code{tooltip} & Brief documentation string for user interfaces.\\
\hline
\code{userlevel} & Integer (0 - 10) representing ease (0) or 
                   difficulty (10) in using this variable, default 0.\\

\hline
\end{tabular}
\label{ag-anno}
\end{table}
